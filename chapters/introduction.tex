\vspace{1in}
{\large
``I can thank the fact that I was born at just the right time. A few years older or younger, I would have missed the opportunity... One might call the period from 1925 onward for a few years the Golden Age of Physics when our basic ideas were developing very rapidly and there was plenty of work for everyone to do."

\hfill -Paul Dirac, J. Robert Oppenheimer Memorial Prize Speech, 1969}
\vspace{1in}

The Standard Model (SM) of Particle Physics, in its current form, is the most (or even the only) widely accepted complete theory of the particle nature of the universe. It provides a mathematical description of the three out of four fundamental forces of nature: Electroweak Force, Strong Force, and the Electromagnetic Force. After its full formulation circa 1975, the model has proved to be extremely successful in predicting natural phenomena at the smallest of scales and highest of energies. Despite that, a few but critical open problems challenge the foundations and completeness of the SM, and beg for theoretical extensions beyond the Standard Model (BSM) that can provide solutions to these problems. Chapter~\ref{chap:theory} outlines the fundamental structure of the SM and lists the open problems that plague it. An extension to the SM in the form of long-lived sterile neutrinos is presented which forms the baselines model used in the analysis presented in this dissertation. The apparatus used to test this model is the ATLAS experiment at the Large Hadron Collider, both of which are described in~\cref{chap:experiment}, along with the techniques used to collect data and stitch the electric signatures into individual physical objects such as particles and their decays. The overview of this analysis is presented in~\cref{chap:ana_overview} with the exact goals we have set out to achieve. The various steps of the analysis method and the statistical structure of extracting the strength of the model are detailed in~\cref{chap:ana_strategy}, followed by the results in~\cref{chap:results}.~\Cref{chap:conclusion} summarizes the inferences of this search, explains its limitations, and lists out possible improvements in the future versions of this adventure.