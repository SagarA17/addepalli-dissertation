\section*{Contributions to This Analysis}
The work presented in this dissertation is a collaborative effort taken up by members of the ATLAS displaced HNL team. As the lead student on this analysis, I have contributed to all the aspects of this search. I have designed, optimized, and validated the secondary vertexing algorithm parameters as described in~\cref{chap:ana_overview}. The algorithm is critical since its the highest point of efficiency loss in the analysis, and ensures that future channel expansions are possible. I have contributed to the background studies and estimation strategies described in detail in~\cref{chap:ana_strategy}, including the implementation and optimization of the discriminant variable $\mathcal{S}$ cut, the implementation of the DV lepton isolation (as well as the close-by-effects), and the region definitions. The systematic uncertainties (see ~\cref{sec:sig_ext}) considered in this analysis are put in place by me. Specifically, I have contributed independently to the standardized use of LRT muons, including their identification WP calibration and uncertainty evaluation. I designed the standard--LRT muon overlap removal strategy which maximized signal retention efficiency. I contributed to the design of the signal extraction strategy using which I performed the fits to obtain the exclusion contours shown in~\cref{chap:results}. I am responsible for maintaining all the data formats used in this analysis, starting from DAOD\_LLP1, analysis ntuples with systematics, and micro-ntuples which contain information per channel for the six channels considered. Lastly, as the editor of the internal documentation, I ensure that all studies pertaining to the analysis are documented well for streamlined internal reviews and future publications.

\section*{Other Contributions to ATLAS}
Apart from this analysis, I have had the opportunity to contribute to multiple projects within ATLAS as its member since 2019.

\textbf{Detector Development}: I was stationed at Brookhaven National Lab from August, 2019 to July, 2021 as a key member of the stave assembly team for the ATLAS Inner Tracker upgrade. I assembled the first stave with 28 modules with the latest chip configuration which helped the stave assembly project pass the ATLAS final design review. I was the lead student in the effort to optimize the quality control criteria for the construction of stave cores, coordinating the design and manufacturing between multiple labs in the UK and US. My final contribution to the project was the development of a new metrology system that combined a camera and a laser to yield the first 3D mapping of staves.

\textbf{Detector Performance}: Working on the search for HNLs gave me the opportunity to join a coordinated effort within ATLAS to provide recommendations on how to use displaced muons for LLP analyses. I lead this activity for ATLAS, developing new methods to calibrate the efficiencies of non-standard displaced muons. I work closely with the tracking and the muon combined performance groups for this activity, and my results are used by all analyses which probe signatures using LRT muons.

\textbf{Physics Analyses}: During my time at BNL, I started working on cross-section measurements of the $H\to WW^*$ process in the vector boson fusion production channel. I started with optimizing the hyperparameters of and training the boosted decision trees used in the analysis to discriminate the signal against background. I took up the task to evaluate the impact of theory modeling systematics on the various processes. Eventually, I stepped into the role of the lead student, optimizing the final binning of the results reported in the analysis and liaising with the journal referees to address comments. I wrote the framework to evaluate the correlations between various differential cross-section measurements using a bootstrapping mechanism, which formed the last results to go into the paper making the analysis complete. My involvement in the $H\to WW^*$ group enabled me to contribute to two other analysis in the same final state but different production mechanism and goals. Furthermore, my expertise in the muon combined performance group allowed me to contribute to the $H\to ZZ^* \to 4\ell$ analysis, where I performed studies to understand the impact of the \texttt{Loose} identification WP on the analysis. This analysis was the first Run-3 ATLAS paper.