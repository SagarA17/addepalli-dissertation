The Standard Model is the most successful theory of particle physics to date. The precise predictions made by the SM have been validated to high degrees of precision by generations of experiments. However, experimental results have also indicated that the SM is incomplete in its current form. Long-Lived Particles are excellent candidates for extensions to the SM since the models predicting them provide explanations to many of the open problems in particle physics. The LHC provides a unique opportunity to look for such particles using the ATLAS detector. 

While the standard ATLAS reconstruction techniques limit the sensitivity reach to look for LLPs, dedicated reconstruction techniques are continuously being developed and improved to use the ATLAS dataset to its full potential in the exploration of LLPs using unique signatures. The analysis presented in this dissertation searches for Heavy Neutral Leptons decaying into a pair of charged leptons which is reconstructed as a secondary vertex in the Inner Detector volume. This analysis is sensitive to a wide range of the HNL mass, $1\text{ GeV}<\mhnl<16\text{ GeV}$ and to HNL-$\nu$ mixing angles as low as $10^{-7}$. The results are a direct improvement on the 2022 ATLAS search~\cite{PhysRevLett.131.061803} using the same final state and dataset, but better reconstruction and analysis techniques. The sensitivity reach in the form of an exclusion contour is significantly expanded from all directions, with the results being statistically limited. No statistically significant deviation from the Standard Model prediction was observed. A paper that will publish these results is currently in preparation.


\section{Outlook}
This analysis features many new tools and techniques, the highlight being an MC-driven shape estimate for the HF background. However, the analysis faces some limitations at various steps, the most significance being the trigger scheme. The displaced leptons typically have lower energies and \pT in the HNL mass range being probed. Since single-lepton triggers have a high \pT threshold (27 GeV), the increase in acceptance from triggering on the displaced leptons is very small. Furthermore, it was found that the analysis was unable to model the HF background in the prompt lepton $\pT<27$~GeV phase space. This multi-fold problem meant that the analysis is limited to using the single-lepton triggers, and were not able to explore ``softer" signatures, which would help expand sensitivity to lower mass and lower couplings HNLs. The DV reconstruction uses ID tracks, which puts a hard cut-off at 300 mm on the radial displacement of the DV. Being limited to this volume also restricted the HNL lifetime reach of the analysis, since the event yield from larger proper lifetime HNLs became excessively small in this volume with the 140.1 fb$^{-1}$ dataset.

This analysis is largely statistically limited, meaning that additional data would significantly improve the sensitivity reach. The HL-LHC program provides the opportunity to collect such a dataset and extend the sensitivity of the analysis. Apart from larger statistics, the sensitivity to lower HNL couplings can also be extended using complementary analysis techniques and final states, a direction that can be explored using the new Run-3 ATLAS dataset. Specifically for lower masses, the search for HNLs decaying semi-leptonically into a lepton and a pion could help add sensitivity from a completely non-overlapping channel. This extension is also natural given the two-track nature of the DV which has already been studied in great detail in this analysis. A more significant extension to the sensitivity can come from increasing the DV acceptance beyond the ID. Specifically, vertex reconstruction using clusters in the calorimeter or MS tracks extends the reach to $\ctau=1-10$~m. Such a reconstruction, however, is more challenging since the calorimeter and the MS don't have the same granularity as the ID in the $\phi$-direction, making it difficult to separate genuine displaced decays from random crossings.